%%%%%%%%%%%%%%%%%%%%%%%%%%%%%%%%%%%%%%%%%
% Beamer Presentation
% LaTeX Template
% Version 1.0 (10/11/12)
%
% This template has been downloaded from:
% http://www.LaTeXTemplates.com
%
% License:
% CC BY-NC-SA 3.0 (http://creativecommons.org/licenses/by-nc-sa/3.0/)
%
%%%%%%%%%%%%%%%%%%%%%%%%%%%%%%%%%%%%%%%%%

%----------------------------------------------------------------------------------------
%	PACKAGES AND THEMES
%----------------------------------------------------------------------------------------

\documentclass{beamer}

\mode<presentation> {

% The Beamer class comes with a number of default slide themes
% which change the colors and layouts of slides. Below this is a list
% of all the themes, uncomment each in turn to see what they look like.

%\usetheme{default}
%\usetheme{AnnArbor}
%\usetheme{Antibes}
%\usetheme{Bergen}
%\usetheme{Berkeley}
%\usetheme{Berlin}
%\usetheme{Boadilla}
%\usetheme{CambridgeUS}
%\usetheme{Copenhagen}
%\usetheme{Darmstadt}
%\usetheme{Dresden}
%\usetheme{Frankfurt}
%\usetheme{Goettingen}
%\usetheme{Hannover}
%\usetheme{Ilmenau}
%\usetheme{JuanLesPins}
%\usetheme{Luebeck}
%\usetheme{Madrid}
%\usetheme{Malmoe}
%\usetheme{Marburg}
%\usetheme{Montpellier}
\usetheme{PaloAlto}
%\usetheme{Pittsburgh}
%\usetheme{Rochester}
%\usetheme{Singapore}
%\usetheme{Szeged}
%\usetheme{Warsaw}

% As well as themes, the Beamer class has a number of color themes
% for any slide theme. Uncomment each of these in turn to see how it
% changes the colors of your current slide theme.

%\usecolortheme{albatross}
%\usecolortheme{beaver}
%\usecolortheme{beetle}
%\usecolortheme{crane}
%\usecolortheme{dolphin}
%\usecolortheme{dove}
%\usecolortheme{fly}
%\usecolortheme{lily}
%\usecolortheme{orchid}
%\usecolortheme{rose}
%\usecolortheme{seagull}
%\usecolortheme{seahorse}
%\usecolortheme{whale}
%\usecolortheme{wolverine}

%\setbeamertemplate{footline} % To remove the footer line in all slides uncomment this line
%\setbeamertemplate{footline}[page number] % To replace the footer line in all slides with a simple slide count uncomment this line

%\setbeamertemplate{navigation symbols}{} % To remove the navigation symbols from the bottom of all slides uncomment this line
}

\usepackage{graphicx} % Allows including images
\usepackage{booktabs} % Allows the use of \toprule, \midrule and \bottomrule in tables

%----------------------------------------------------------------------------------------
%	TITLE PAGE
%----------------------------------------------------------------------------------------

\title[Short title]{Hyperion Architecture Investigation} % The short title appears at the bottom of every slide, the full title is only on the title page

\author{Diego V.} % Your name
%\institute[UCLA] % Your institution as it will appear on the bottom of every slide, may be shorthand to save space
%{
%University of California \\ % Your institution for the title page
%\medskip
%\textit{john@smith.com} % Your email address
%}
\date{\today} % Date, can be changed to a custom date

\begin{document}

\begin{frame}
\titlepage % Print the title page as the first slide
\end{frame}

\begin{frame}
\frametitle{Overview} % Table of contents slide, comment this block out to remove it
\tableofcontents % Throughout your presentation, if you choose to use \section{} and \subsection{} commands, these will automatically be printed on this slide as an overview of your presentation
\end{frame}

%----------------------------------------------------------------------------------------
%	PRESENTATION SLIDES
%----------------------------------------------------------------------------------------

%------------------------------------------------
\section{Preliminar Overview} % Sections can be created in order to organize your presentation into discrete blocks, all sections and subsections are automatically printed in the table of contents as an overview of the talk
%------------------------------------------------

\begin{frame}
\frametitle{What is Hyperion?}
\begin{itemize}
\item It is an multiprocess discrete event based, SystemVerilog Compiler and Simulator.
\item It publishes events over time (See Standard section 9.4.2 for definition of events).
\item Every change in state of a net or variable in the system description being simulated is considered an update event (Section 4.3)
\item The evaluation of a process is also an event, known as an evaluation event.
\end{itemize}
\end{frame}

%------------------------------------------------

\begin{frame}
\frametitle{General Design Thoughs}
\begin{itemize}
\item The simulator follows an Observer Design pattern.
\item Each process subscribes to zero or more events (sentivity list)
\item Only observed nets are evaluated for event changes (for performance reasons, we will not evaluate unconnected logic )
\item Individual events are published during corresponding timeslot region.
\end{itemize}
\end{frame}

%------------------------------------------------

\begin{frame}
\frametitle{General Design Thoughs}
\begin{itemize}
\item We will use google Go because if the easy of use of IPC and multiprocess handling
\item We will rely on gochannels to pass pointers to events to each event subscriber
\item  We will heavily rely on dictionaries for ~O(1) updates
\item Each process can pontentialy be a gorutine (legend hass it that goe can handle millions of gorutines http://tleyden.github.io/blog/2014/10/30/goroutines-vs-threads/)
\end{itemize}
\end{frame}

%------------------------------------------------

\begin{frame}
\frametitle{Simulator Hierarchies}
\begin{itemize}
\item There are no block or hierarchies per-se.
\item Verilog 'modules' are not processes (per standard)
\item On the other hand, each single *module port* is a process (Implicit continous assignement as per standard section 4.9.6)
\item Each symbol in the simulator (including processes) has a unique key. (has the full path prepended to the symbol dictionary, this is then transformed to 128bit hash key)	
\end{itemize}
\end{frame}

%------------------------------------------------

\begin{frame}
\Huge{\centerline{The End}}
\end{frame}

%----------------------------------------------------------------------------------------

\end{document} 